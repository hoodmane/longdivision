
\expandafter \ifx \csname longdivision-loaded\endcsname \relax \else
    \endinput
\fi
\expandafter\def\csname longdivision-loaded\endcsname{}

\input expl3-generic

\ExplSyntaxOn

%%
%% The linked list
%%

% This token list just stores a reference to the first entry in the linked list
\tl_new:N \l__longdiv_linkedlist_tl
\tl_set:Nn\l__longdiv_linkedlist_tl{ \longdiv_linkedlist_next:n { 0 } }
\int_new:N \l__longdiv_linkedlist_length_int

% Set the next entry to be a no-op so that when expanded the last "null pointer" just disappears
\cs_new:Nn \longdiv_linkedlist_set_next_do_nothing: {
    \cs_set_eq:cN { longdiv_linkedlist ~ \int_use:N \l__longdiv_linkedlist_length_int } \prg_do_nothing:
}
\longdiv_linkedlist_set_next_do_nothing:

% "pointer" to next element (argument is the element's id)
\cs_new:Nn \longdiv_linkedlist_next:n { \use:c { longdiv_linkedlist ~ #1 } }

\cs_new:Nn \longdiv_linkedlist_add:n {
    \cs_set:cpx { longdiv_linkedlist ~ \int_use:N \l__longdiv_linkedlist_length_int}{
        \exp_not:n{#1} \exp_not:N \longdiv_linkedlist_next:n { \int_eval:n { \l__longdiv_linkedlist_length_int + 1} }
    }
    \int_incr:N \l__longdiv_linkedlist_length_int
    \longdiv_linkedlist_set_next_do_nothing:
}
\cs_generate_variant:Nn \longdiv_linkedlist_add:n {f}

% The easy implementation of these next two commands is why I chose the "linked list" format

% Delete last element of list.
\cs_new:Nn \longdiv_linkedlist_remove_tail: {
    \int_decr:N \l__longdiv_linkedlist_length_int
    \longdiv_linkedlist_set_next_do_nothing:
}

% #1 -- pointer to an element of the list
% Puts an overline over the rest of the list starting at a certain position. For "repeating decimals".
\cs_new:Nn \longdiv_linkedlist_put_overline:n {
    \cs_set:cpx { longdiv_linkedlist ~ #1 }{
        $\overline { \exp_not:f { \use:c { longdiv_linkedlist ~ #1 } } }$
    }
}


%%
%% Entry points
%%

% Some annoying stuff to ensure that errors refer to \longdivision. Also note that the commands that take arguments are called _nopar.
% The only thing that happens here is the optional argument is put into \l__longdiv_max_extra_digits_int and we rescan to
% ignore spaces in the input.
\cs_new:Nn \longdivision_call:N { \group_begin:  \cs_set_eq:NN \longdivision #1 \exp_after:wN \group_end: \longdivision }
\cs_new:Npn \longdivision {
    \group_begin:
    \peek_meaning:NTF [ { \longdivision_call:N \longdivision_opt:w } { \longdivision_call:N\longdivision_main:nn }
}
\cs_new_nopar:Npn \longdivision_opt:w [ #1 ] {
    \longdivision_handle_oarg:n { #1 } {
        \longdivision_call:N \longdivision_main:nn
    }
}
\cs_new_nopar:Nn \longdivision_main:nn {
    \tl_rescan:nn { \ExplSyntaxOn } { \longdiv_start:nn { #1 } { #2 } }
    \group_end:
}


\cs_new:Nn \longdivision_handle_oarg:n {
    \afterassignment \longdivision_handle_oarg_aux:w
    \l__longdiv_max_extra_digits_int = 0 #1 \scan_stop: { #1 }
}

\cs_new:Npn \longdivision_handle_oarg_aux:w #1 \scan_stop: #2 {
    \tl_if_empty:nTF { #1 } {
        \use:n
    }{
        \msg_error:nnn { longdivision } { max_extra_digits_invalid } { #2 }
        \use_none:n
    }
}


% Same as \longdiv[0]{#1}{#2}.
\cs_new_nopar:Npn \intlongdivision #1 #2 {
    \group_begin:
    \int_set:Nn \l__longdiv_max_extra_digits_int { 0 }
    \tl_rescan:nn { \ExplSyntaxOn } { \longdiv_start:nn { #1 } { #2 } }
    \group_end:
}

% Check input is valid then enter main loop.
% We use etex \numexpr to ensure that the dividend has no unnecessary leading zeroes and doesn't begin with a decimal point.
% Note that \int_eval:n wouldn't work here because it inserts a "\relax" token that would not get eaten by \numexpr if
% #1 contains a decimal point. This "\relax" causes trouble for the division main loop.
\cs_new:Nn \longdiv_start:nn {
    \longdiv_check_dividend:n { #1 }
    \longdiv_check_divisor:n { #2 }
    % Second copy of #1 is eaten by \longdiv_typeset:nnn to print the dividend
    \exp_args:Nnnff \longdiv_get_new_digit:nnn { } { #2 } { \the\numexpr 0#1 } { \the\numexpr 0#1  }
    \longdiv_break_point: { #1 } { #2 }
}

\cs_set_eq:NN \longdiv_break_point: \use_none:nn

%%
%% Input checkers
%%

% Parse through the dividend token by token
% Check that every token is a digit with the exception of at most one .
\cs_new:Nn \longdiv_check_dividend:n {
    \longdiv_check_dividend_before_point:N #1 \q_stop
}

\cs_new:Nn \longdiv_check_dividend_before_point:N {
    \token_if_eq_meaning:NNF #1 \q_stop {
        \token_if_eq_meaning:NNTF #1 . {
            \longdiv_check_dividend_seen_point:N
        }{
            \longdiv_check_dividend_isdigit:N #1
            \longdiv_check_dividend_before_point:N
        }
    }
}

\cs_new:Nn \longdiv_check_dividend_seen_point:N {
    \token_if_eq_meaning:NNF #1 \q_stop {
        \longdiv_check_dividend_isdigit:N #1
        \longdiv_check_dividend_seen_point:N
    }
}

\cs_new:Nn \longdiv_check_dividend_isdigit:N {
    \bool_if:nF { \token_if_eq_meaning_p:NN #1 0
      || \token_if_eq_meaning_p:NN #1 1 || \token_if_eq_meaning_p:NN #1 2 || \token_if_eq_meaning_p:NN #1 3
      || \token_if_eq_meaning_p:NN #1 4 || \token_if_eq_meaning_p:NN #1 5 || \token_if_eq_meaning_p:NN #1 6
      || \token_if_eq_meaning_p:NN #1 7 || \token_if_eq_meaning_p:NN #1 8 || \token_if_eq_meaning_p:NN #1 9
    }{
        \longdiv_error:nwnn { dividend_invalid }
    }
}

% Check that there is no ., that it is at most 8 digits, and that the entire argument can get assigned to a count variable
% There's no way to do this last check in expl3, so I use plaintex \newcount, \afterassignment, and \l_tmpa_int =.
\cs_new:Nn \longdiv_check_divisor:n {
    \tl_if_in:nnT { #1 } { . } {
        \longdiv_error:nwnn { divisor_not_int }
    }
    % We have to do the length check before the "validity" check because the "validity" check makes an assignment
    % which throws a low level error if the number to be assigned is too large.
    \int_compare:nNnF { \tl_count:n { #1 } } < \c_nine {
        \longdiv_error:nwnn { divisor_too_large }
    }
    % Idea here: if #1 is a valid number, \l_tmpa_int = 0#1 will absorb all of it.
    % So if there's any left, throw an error. Leading zero ensures that it fails on -1 and
    % that if #1 starts with some other nondigit character that it won't cause
    % "Missing number, treated as zero."
    \afterassignment \longdiv_check_divisor_aux:w
    \l_tmpa_int = 0 #1 \scan_stop:
}

\cs_new:Npn \longdiv_check_divisor_aux:w #1 \scan_stop: {
    \tl_if_empty:nF { #1 } {
        \longdiv_error:nwnn { divisor_invalid }
    }
    \int_compare:nNnT \l_tmpa_int = \c_zero {
        \longdiv_error:nwnn { divisor_zero }
    }
}

% Absorb up to break_point to gracefully quit out of the macro
\cs_new:Npn \longdiv_error:nwnn #1 #2 \longdiv_break_point: {
    \msg_error:nnnn { longdivision } { #1 }
}

% Errors:
\msg_new:nnn { longdivision } { dividend_invalid } { Dividend ~ '#1' ~ not ~ a ~ valid ~ positive ~ decimal ~ number ~ (\msg_line_context:).}
\msg_new:nnn { longdivision } { divisor_too_large }
    { Divisor ~ '#2' ~ is ~ too ~ large ~ (\msg_line_context:). ~ It ~ has ~ \tl_count:n { #2 } ~ digits, ~ but ~ divisors ~ can ~ be ~ at ~ most ~ 9 ~ digits ~ long. }
\msg_new:nnn { longdivision } { divisor_not_int } { Divisor ~ '#2' ~ is ~ not ~ an ~ integer ~ (\msg_line_context:). }
\msg_new:nnn { longdivision } { divisor_invalid } { Divisor ~ '#2' ~ is ~ not ~ a ~ valid ~ positive ~ integer ~ (\msg_line_context:). }

\msg_new:nnn { longdivision } { max_extra_digits_invalid } { Optional ~ argument ~ '#1' ~ is ~ not ~ a ~ valid  ~ nonnegative ~ integer ~ (\msg_line_context:) }

% Warnings:
\msg_new:nnn { longdivision } { work_stopped_early } { The ~ work ~ display ~ stopped ~ early ~ to ~ avoid ~ running ~ off ~ the ~ page ~ (\msg_line_context:). }
\msg_new:nnn { longdivision } { division_stopped_early } { The ~ division ~ stopped ~ early ~ to ~ avoid ~ running ~ off ~ the ~ page ~ (\msg_line_context:).}
\msg_new:nnn { longdivision } { no_division_occurred }
    { Either ~ the ~ dividend ~ was ~ zero ~ or ~ you ~ used ~ \token_to_str:N \intlongdiv \space and ~ the ~ dividend ~ was ~less ~ than ~ the ~ divisor. ~
      This ~ isn't ~ a ~ big ~ deal, ~ but ~ the ~ result ~ probably ~ looks ~ silly. }



%%
%% Division
%%

% Core registers
\bool_new:N \l__longdiv_seen_point_bool
\bool_new:N \l__longdiv_seen_digit_bool
\int_new:N \l__longdiv_quotient_int
\int_new:N \l__longdiv_position_int
\int_new:N \l__longdiv_point_digit_int
\tl_new:N \l__longdiv_extra_places_tl % This is for storing extra zeroes appended to the dividend

% These are used to make sure division doesn't run off the page.
\int_new:N \l__longdiv_extra_digits_int
\int_new:N \l__longdiv_max_extra_digits_int
\int_set:Nn \l__longdiv_max_extra_digits_int { 100 } % Infinite (just needs to be greater than max_total_digits_int and max_display_divisions_int)
\int_const:Nn \c__longdiv_max_total_digits_int { 60 }
\int_const:Nn \c__longdiv_max_display_divisions_int { 20 }
\int_new:N \l__longdiv_display_divisions_int

% #1 -- remainder
% #2 -- divisor
% #3 -- rest of divits
\cs_new:Nn \longdiv_get_new_digit:nnn {
    \tl_if_empty:nTF { #3 } { % Are we out of digits?
        % If we haven't hit the decimal point add it to the quotient and dividend
        % Set seen_digit false so that we can remove the decimal point later if it divided evenly or we used \intlongdiv
        \bool_if:NF \l__longdiv_seen_point_bool {
            \longdiv_add_point: %
            \tl_set:Nn \l__longdiv_extra_places_tl { . }
            \bool_set_false:N \l__longdiv_seen_digit_bool
        }
        \longdiv_divide_no_more_digits:nn { #1 } { #2 }
    }{
        \longdiv_get_new_digit_aux:nnw { #1 } { #2 } #3;
    }
}
\cs_generate_variant:Nn \longdiv_get_new_digit:nnn {xnn}

\cs_new:Npn \longdiv_get_new_digit_aux:nnw #1 #2 #3 #4;{
    \token_if_eq_meaning:NNTF #3 . {
        \longdiv_add_point:
        \bool_set_true:N \l__longdiv_seen_digit_bool % Prevent this decimal point from being removed later
        \longdiv_get_new_digit:nnn { #1 } { #2 } { #4 }
    }{
        \longdiv_divide:nn { #1 #3 } { #2 } { #4 }
    }
}

% Adds a decimal point, with a leading 0 if necessary, and records the current position in \l__longdiv_point_digit_int
\cs_new:Nn \longdiv_add_point: {
    \bool_set_true:N \l__longdiv_seen_point_bool
    \bool_if:NTF \l__longdiv_seen_digit_bool {
        \longdiv_linkedlist_add:n { . }
    }{
        \longdiv_linkedlist_add:n { 0. } % Add a leading zero

    }
    \int_set_eq:NN \l__longdiv_point_digit_int \l__longdiv_position_int % Record the position of the point
}

% Divide when we still have more digits.
% #1 -- thing to divide
% #2 -- divisor
% Finds the quotient, adds it to the linked list and to the work token list then recurses.
\cs_new:Nn \longdiv_divide:nn {
    %\tl_show:n{#1}
    \int_set:Nn \l__longdiv_quotient_int { \int_div_truncate:nn { #1 } { #2 } }
    \bool_if:nF {
        \int_compare_p:nNn \l__longdiv_quotient_int = \c_zero % If the quotient was zero, we might not have to print it
        && !\l__longdiv_seen_digit_bool % If no other digits have been printed
        && !\l__longdiv_seen_point_bool % And we are before the decimal point
    }{ % Otherwise print it and record that we've seen a digit (all further 0's must be printed)
        \bool_set_true:N \l__longdiv_seen_digit_bool
        \longdiv_linkedlist_add:f { \int_use:N \l__longdiv_quotient_int }
    }
    \int_incr:N \l__longdiv_position_int
    \longdiv_divide_record:nn{ #1 }{ #2 }
    \longdiv_get_new_digit:xnn { \longdiv_remainder:nn { #1 } { #2 } } { #2 }
}


% Divide when we are out of digits.
% #1 -- remainder from last time (we will add a zero to the end)
% #2 -- divisor
% This case is more complicated because we have to check for repeated remainders, and whether to stop
% though we are certainly after the decimal point so we don't need to check whether we need to print 0's.
\cs_new:Nn \longdiv_divide_no_more_digits:nn {
    % If we've seen this remainder before, we're done. Use the appropriate command
    % to insert the overline, and then typeset everything
    \cs_if_exist_use:cTF { longdiv_remainders ~ \int_eval:n { #1 } }{ % \int_eval:n to remove leading zero
        \longdiv_typeset:nnn { #1 } { #2 }
    }{
        \bool_if:nTF {
              \int_compare_p:nNn \l__longdiv_extra_digits_int = \l__longdiv_max_extra_digits_int
            ||\int_compare_p:nNn \l__longdiv_position_int = \c__longdiv_max_total_digits_int
        }{
            \int_compare:nNnT \l__longdiv_position_int = \c__longdiv_max_total_digits_int {
                \msg_warning:nn { longdivision } { division_stopped_early }
            }
            \longdiv_typeset:nnn{#1}{#2}
        }{
            % Otherwise, record that we've seen this remainder and the position we're in
            % In case this is the first digit of the repeated part
            \cs_set:cpx { longdiv_remainders ~ \int_eval:n { #1 } }{ % \int_eval:n to remove leading zero
                \exp_not:N \longdiv_linkedlist_put_overline:n { \int_use:N \l__longdiv_linkedlist_length_int }
            }
            % Now we have to use #10 everywhere
            \int_set:Nn \l__longdiv_quotient_int { \int_div_truncate:nn { #10 } { #2 } }
            \longdiv_linkedlist_add:f { \int_use:N \l__longdiv_quotient_int }
            \bool_set_true:N \l__longdiv_seen_digit_bool % We've seen a digit after the decimal point, don't need to remove it
            \int_incr:N \l__longdiv_position_int
            \int_incr:N \l__longdiv_extra_digits_int
            \tl_set:Nx \l__longdiv_extra_places_tl { \l__longdiv_extra_places_tl 0 } % Store an extra 0 appended to the dividend
            \longdiv_divide_record:nn{#10}{#2}
            \longdiv_divide_no_more_digits:xn { \longdiv_remainder:nn { #10 } { #2 } } { #2 }
        }
    }
}
\cs_generate_variant:Nn \longdiv_divide_no_more_digits:nn {xn}

% Whenever we see the remainder 0, we're done, and we don't have to put an overline.
\cs_set:cpn { longdiv_remainders ~ 0}{}

% This command checks if the quotient was zero, and if so preserves the leading zero by avoiding \int_eval:n
% This is so that e.g, \longdiv{14.1}{7} doesn't screw up
\cs_new:Nn \longdiv_remainder:nn {
    \int_compare:nNnTF \l__longdiv_quotient_int = \c_zero
        { #1 }
        { \int_eval:n { #1 - \l__longdiv_quotient_int * #2 } }
}


% We're going to store the "work" for the long division in this tl as a series of triples:
% #1 -- number of digits we've processed so far (for positioning subtractions and determining if point should be added)
% #2 -- old remainder (thing to subtract from)
% #3 -- quotient * divisor (thing to subtract)
\tl_new:N \l__longdiv_work_tl
\cs_new:Nn \longdiv_divide_record:nn{
    \int_compare:nNnTF \l__longdiv_display_divisions_int < \c__longdiv_max_display_divisions_int {
        \int_compare:nNnF \l__longdiv_quotient_int = \c_zero { % If the quotient was zero, nothing needs to be typeset
            \tl_set:Nx \l__longdiv_work_tl {
                \l__longdiv_work_tl
                    { \int_use:N \l__longdiv_position_int } { #1 } { \int_eval:n { \l__longdiv_quotient_int * #2 } }
            }
            \int_incr:N \l__longdiv_display_divisions_int
        }
    }{
        \int_compare:nNnT \l__longdiv_display_divisions_int = \c__longdiv_max_display_divisions_int {
            \int_compare:nNnF \l__longdiv_quotient_int = \c_zero {
                \tl_set:Nx \l__longdiv_work_tl {
                    \l__longdiv_work_tl
                        { \int_use:N \l__longdiv_position_int } { #1 } { \int_eval:n { \l__longdiv_quotient_int * #2 } }
                        \exp_not:N \longdiv_typeset_work_last:nn { \int_use:N \l__longdiv_position_int } { \int_eval:n { #1 - \l__longdiv_quotient_int * #2 } }
                }
                \int_incr:N \l__longdiv_display_divisions_int
                \msg_warning:nn { longdivision } { work_stopped_early }
            }
        }
    }
}

%%
%% Typesetting
%%


% \l__longdiv_linkedlist_tl -- quotient
% #1 -- remainder
% #2 -- divisor
% #3 -- dividend
\cs_new:Nn \longdiv_typeset:nnn {
    % If we haven't seen any new digits since adding a terminal decimal point, delete it.
    \bool_if:NF \l__longdiv_seen_digit_bool { \longdiv_linkedlist_remove_tail: }

    $#2\,$\begin{tabular}[b]{@{}r@{}}
    \int_compare:nNnTF \l__longdiv_linkedlist_length_int = \c_zero
        { 0 }
        { \l__longdiv_linkedlist_tl }
    \\\hline
    \big)\begin{tabular}[t]{@{}l@{}}
    % Likewise we don't want to add a trailing decimal point to the dividend if it divided evenly
    #3 \bool_if:NT \l__longdiv_seen_digit_bool { \l__longdiv_extra_places_tl } \\
    \longdiv_typeset_work:n { #1 }
    \end{tabular}
    \end{tabular}
}

% Iterate through the division "work" and typeset it
\cs_new:Nn \longdiv_typeset_work:n {
    \tl_if_empty:NTF \l__longdiv_work_tl {
        \msg_warning:nn { longdivision } { no_division_occurred }
    }{
        \exp_after:wN \longdiv_typeset_work_first:nnn \l__longdiv_work_tl
        \int_compare:nNnT \l__longdiv_display_divisions_int < \c__longdiv_max_display_divisions_int {
            \exp_args:No \longdiv_typeset_work_last:nn { \int_use:N \l__longdiv_position_int } { #1 }
        }
    }
}

% #1 -- digits in to the right side of the numbers we are writing
% #2 -- remainder from last time with new digits added to the right
% #3 -- quotient * divisor
% _first only typesets quotient * divisor and the line
% _rest typesets result from last time, quotient * divisor and the line
% _last only typesets the remainder from last time
\cs_new:Nn \longdiv_typeset_work_first:nnn {
    \longdiv_typeset_setwidth:n { #1 }
    \hspace{\g_tmpa_dim}
    \llap { \longdiv_insert_point_ifneeded:nn { #1 } { #3 } }
    \\\longdiv_rule:nn{#1}{#3}
    \peek_meaning:NT \bgroup {
        \longdiv_typeset_work_rest:nnn
    }
}

\cs_new:Nn \longdiv_typeset_work_rest:nnn {
    \longdiv_typeset_setwidth:n { #1 }
    \hspace{\g_tmpa_dim}
    \llap { \longdiv_insert_point_ifneeded:nn { #1 } { #2 } }
    \\
    \hspace{\g_tmpa_dim}
    \llap { \longdiv_insert_point_ifneeded:nn { #1 } { #3 } }
    \\\longdiv_rule:nn{#1}{#3}
    \peek_meaning:NT \bgroup {
        \longdiv_typeset_work_rest:nnn
    }
}

% #1 -- digits in to the right side of the numbers we are writing
% #2 -- remainder from last time with new digits added to the right
\cs_new:Npn \longdiv_typeset_work_last:nn #1 #2 {
    \longdiv_typeset_setwidth:n { #1 }
    \hspace{\g_tmpa_dim}
    \llap { \longdiv_insert_point_ifneeded:nn { #1 } { #2 } }
}

% Set \g_tmpa_dim equal to digitwidth * number of digits
% If we are past the decimal point, add  \c__longdiv_pointwidth_dim
\cs_new:Nn \longdiv_typeset_setwidth:n {
    \dim_gset:Nn \g_tmpa_dim { #1\c__longdiv_digitwidth_dim }
    \int_compare:nNnT \l__longdiv_point_digit_int < {#1} {
        \dim_gadd:Nn \g_tmpa_dim \c__longdiv_pointwidth_dim
    }
}

% If the number ends after the decimal point ( #1 > \l__longdiv_point_digit_int )
% and start before it ( #1 - length(#2) < \l__longdiv_point_digit_int) insert a
% decimal point in the appropriate position of #2. Otherwise just return #2
\cs_new:Nn \longdiv_insert_point_ifneeded:nn {
    \bool_if:nTF {
          \int_compare_p:nNn  { #1 }                      > \l__longdiv_point_digit_int
        && \int_compare_p:nNn { #1 - \tl_count:n { #2 } } < \l__longdiv_point_digit_int
    }{
        \longdiv_insert_point:nn {\int_eval:n{\l__longdiv_point_digit_int - #1 + \tl_count:n { #2 }}} { #2 }
    }{
        #2
    }
}

% Walk #1 digits across #2 and then insert a decimal point.
\cs_new:Nn \longdiv_insert_point:nn {
    \longdiv_insert_point_aux:oN { #1 } #2
}

\cs_new:Nn \longdiv_insert_point_aux:nN {
    \int_compare:nNnTF { #1 } = \c_zero {
        .#2
    }{
        #2 \longdiv_insert_point_aux:oN { \int_eval:n { #1 - 1 } }
    }
}
\cs_generate_variant:Nn \longdiv_insert_point_aux:nN {oN}


% Okay, this is another section where we are adulterated with plaintex stuff.
% It would be easy to reimplement \settowidth, but \hrule and \noalign have no
% expl3 name anyways. Since I only use these dim variables with \settowidth, I declare
% them with \newdimen rather than \dim_new:N

\cs_new:Nn \longdiv_settowidth:Nn {
    \hbox_set:Nn \l_tmpa_box { #2 }
    \dim_set:Nn #1 { \box_wd:N \l_tmpa_box }
}

\dim_new:N \c__longdiv_digitwidth_dim
\longdiv_settowidth:Nn \c__longdiv_digitwidth_dim { 0 }
\dim_new:N \c__longdiv_pointwidth_dim
\longdiv_settowidth:Nn \c__longdiv_pointwidth_dim { . }

\cs_new:Nn \longdiv_rule:nn {
    \noalign {
        \longdiv_settowidth:Nn \l_tmpa_dim { #2 }
        % Check whether the decimal point occurred in the middle of the current number
        % because if so, it's longer by pointwidth.
        \bool_if:nT {
              \int_compare_p:nNn  { #1 }                      > \l__longdiv_point_digit_int
            && \int_compare_p:nNn { #1 - \tl_count:n { #2 } } < \l__longdiv_point_digit_int
        }{
            \dim_add:Nn \l_tmpa_dim \c__longdiv_pointwidth_dim
        }
        \box_move_right:nn { \g_tmpa_dim - \l_tmpa_dim } {
            \vbox:n { \hrule width \l_tmpa_dim }
        }
    }
}

\ExplSyntaxOff
